\section{Math Stuff}

% Demonstrate inline math - https://www.overleaf.com/learn/latex/Mathematical_expressions
% $i$ or \(i\)
Inline math allows us to state $e^{i\pi} + 1 = 0$ without interruption.
Display math is useful for important equations etc., which may in some cases requires bigger notation, such as matrices and case distinctions.
% Demonstrate display math
% \[ D \] preferable to $$ D $$
\[
	1
	= \det I_n
% Parenthesized matrix - https://www.overleaf.com/learn/latex/Matrices
	= \det\begin{pmatrix}
	1      & 0 & \cdots & 0      \\
	0      & 1 &        & 0      \\
	\vdots &   & \ddots & \vdots \\
	0      & 0 & \cdots & 1      \\
	\end{pmatrix}
% Custom bracket size
% Overbraces with `\overbrace{x}^{y}`
% Text in math with `\text` and nested math with $$
% Stacking symbols with `\stackrel`
% Case distinctions with `\begin{cases}`
	= \det\Big|\overbrace{ (\delta_{ij})_{ij} }^{\in\C^{n\times n}}\Big|
	\quad\text{where}\ \delta_{ij}
	:= \begin{cases} % https://math.stackexchange.com/questions/944757/whats-the-most-right-symbol-to-use-for-defined-to-be-equal-to
		1 & \text{if $i = j$} \\ 0 & \text{else.}
	\end{cases}
\]

We can refer to labeled equations, such as equation \eqref{eq:1}.

\textit{Given $\delta(P) = \max_{1\leq i\leq n}\{x_i-x_{i-1}\}$ for some partition $P = \{x_0,\dots,x_n\}\subseteq I$.
The function $f:I\to\R$ is integrable iff the following limit exists:}
% Demonstrate a math equation
\begin{equation}\label{eq:1}
	\int_a^b f(x) \,\de x \stackrel{\mathrm{def.}}{=} \lim_{\delta(P)\to0} \sum_{i=0}^n f(\xi_i)\cdot(x_i - x_{i-1})
\end{equation}

A single, long equation can be split as follows.
% Demonstrate breakable equation
% Star (*) to prevent labeling
\begin{multline*}
	f(z)
	= \sum_{n=0}^\infty \frac{f^{(n)}(c) (z-c)^n}{n!}
	= f(c) + f'(c)(z-c) + \frac{f''(c)}{2!}(z-c)^2 \\
	+ \frac{f^{(3)}(c)}{3!}(z-c)^3 + \frac{f^{(4)}(c)}{4!}(z-c)^4 + \dots
\end{multline*}

Typeset aligned mathematics when several splits are necessary;
It must be remarked that this is exceedingly useful.
\def\QED{\blacksquare}%{\;\rule{1.5ex}{1.5ex}\;}
\textit{Given $a*\widehat{a} = e$, consider $\widehat{a}*a \stackrel?= e$.}
% Demonstrate multiline math
\begin{align*}
	\widehat{a}*a
	&= (\widehat{a}*a)*e &\text{Right neutral element.} \\
	&=(\widehat{a}*a)*(\widehat{a}*\widehat{\widehat{a}}) &\text{Right inverse element.} \\
	&= \widehat{a}*(a*(\widehat{a}*\widehat{\widehat{a}})) &\text{Associativity.} \\
	&= \widehat{a}*((a*\widehat{a})*\widehat{\widehat{a}}) &\text{Associativity.} \\
	&= \widehat{a}*(e*\widehat{\widehat{a}}) &\text{Right inverse element.} \\
	&= (\widehat{a}*e)*\widehat{\widehat{a}} &\text{Associativity.} \\
	&= \widehat{a}*\widehat{\widehat{a}} &\text{Right neutral element.} \\
	&= e &\text{Right inverse element.}  &\QED \\
\end{align*}

% Demonstrate some other math commands
Finally, other notable miscellany include
$$
\sqrt[n]{xyz}\ \widetilde{xyz}\ \widehat{xyz}\ \overline{ABC}\ \underline{ABC}\ \overrightarrow{uvw}\ \overleftarrow{uvw}\ \overbrace{\alpha\beta\gamma}^{\text{hi}}\ \underbrace{\alpha\beta\gamma}_{\text{hello}}
$$
\& $\imath$ and $\jmath$ can be used to produce custom bedecked versions like $\hat\imath$ and $\vec\jmath$.

\bigskip
\todo[\href{https://www.ctan.org/pkg/bussproofs}{\texttt{bussproofs.}}]