\raisebox{-0.4ex}{\huge V}\resizebox{0pt}{0pt}{on\space Franz\space Kafka\space -\space v}or dem Gesetz steht ein Türhüter. Zu diesem Türhüter kommt ein Mann vom Lande und bittet um Eintritt in das Gesetz. Aber der Türhüter sagt, dass er ihm jetzt den Eintritt nicht gewähren könne. Der Mann überlegt und fragt dann, ob er also später werde eintreten dürfen. „Es ist möglich,“ sagt der Türhüter, „jetzt aber nicht.“ Da das Tor zum Gesetz offen steht wie immer und der Türhüter beiseite tritt, bückt sich der Mann, um durch das Tor in das Innere zu sehen. Als der Türhüter das merkt, lacht er und sagt: „Wenn es dich so lockt, versuche es doch trotz meines Verbotes hineinzugehen. Merke aber: Ich bin mächtig. Und ich bin nur der unterste Türhüter. Von Saal zu Saal stehen aber Türhüter, einer mächtiger als der andere. Schon den Anblick des Dritten kann nicht einmal ich mehr ertragen.“ Solche Schwierigkeiten hat der Mann vom Lande nicht erwartet; das Gesetz soll doch jedem und immer zugänglich sein, denkt er, aber als er jetzt den Türhüter in seinem Pelzmantel genauer ansieht, seine grosse Spitznase, den langen, dünnen, schwarzen tartarischen Bart, entschliesst er sich doch lieber zu warten, bis er die Erlaubnis zum Eintritt bekommt. Der Türhüter gibt ihm einen Schemel und lässt ihn seitwärts von der Tür sich niedersetzen. Dort sitzt er Tage und Jahre. Er macht viele Versuche eingelassen zu werden und ermüdet den Türhüter durch seine Bitten. Der Türhüter stellt öfters kleine Verhöre mit ihm an, fragt ihn über seine Heimat aus und nach vielem andern, es sind aber teilnahmslose Fragen, wie sie grosse Herren stellen, und zum Schlusse sagt er ihm immer wieder, dass er ihn noch nicht einlassen könne. Der Mann, der sich für seine Reise mit vielem ausgerüstet hat, verwendet alles, und sei es noch so wertvoll, um den Türhüter zu bestechen. Dieser nimmt zwar alles an, aber sagt dabei: „Ich nehme es nur an, damit du nicht glaubst, etwas versäumt zu haben.“ Während der vielen Jahre beobachtet der Mann den Türhüter fast ununterbrochen. Er vergisst die andern Türhüter und dieser erste scheint ihm das einzige Hindernis für den Eintritt in das Gesetz. Er verflucht den unglücklichen Zufall, in den ersten Jahren rücksichtslos und laut, später als er alt wird, brummt er nur noch vor sich hin. Er wird kindisch und da er in dem jahrelangen Studium des Türhüters auch die Flöhe in seinem Pelzkragen erkannt hat, bittet er auch die Flöhe ihm zu helfen und den Türhüter umzustimmen. Schliesslich wird sein Augenlicht schwach und er weiss nicht, ob es um ihn wirklich dunkler wird oder ob ihn nur seine Augen täuschen. Wohl aber erkennt er jetzt im Dunkel einen Glanz, der unverlöschlich aus der Türe des Gesetzes bricht. Nun lebt er nicht mehr lange. Vor seinem Tode sammeln sich in seinem Kopfe alle Erfahrungen der ganzen Zeit zu einer Frage, die er bisher an den Türhüter noch nicht gestellt hat. Er winkt ihm zu, da er seinen erstarrenden Körper nicht mehr aufrichten kann. Der Türhüter muss sich tief zu ihm hinunterneigen, denn der Grössenunterschied hat sich sehr zu ungunsten des Mannes verändert. „Was willst du denn jetzt noch wissen?“ fragt der Türhüter, „du bist unersättlich.“ „Alle streben doch nach dem Gesetz,“ sagt der Mann, „wieso kommt es, dass in den vielen Jahren niemand ausser mir Einlass verlangt hat?“ Der Türhüter erkennt, dass der Mann schon an seinem Ende ist und, um sein vergehendes Gehör noch zu erreichen, brüllt er ihn an: „Hier konnte niemand sonst Einlass erhalten, denn dieser Eingang war nur für dich bestimmt. Ich gehe jetzt und schliesse ihn.“
